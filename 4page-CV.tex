% update chapters
% update NSF grant
%________________________________________________________________________________________
% @brief    LaTeX2e Resume for C. Titus Brown
% @author   Titus Brown
% @date     January 2008
% @info     Based on Latex Resume Template by Chris Paciorek 
%           http://www.biostat.harvard.edu/~paciorek/


%________________________________________________________________________________________
\documentclass[margin,line]{resume}

\usepackage{hyperref}
\usepackage{color}
\hypersetup{urlcolor=blue}

\begin{document}
\name{\Large C. Titus Brown}
\pagenumbering{arabic}
\pagestyle{plain}


\begin{resume}


    %____________________________________________________________________________________

    {\small (As of January 2018.)}

    % Education
    \section{\mysidestyle Education}

    {\bf Reed College}, Portland, OR; Mathematics; B.A., 1997

    \vspace{2mm}

    {\bf California Institute of Technology}, Davidson Lab (graduate student);
\\
Developmental Biology; PhD., 2007

    \vspace{2mm}

    {\bf California Institute of Technology}, Bronner-Fraser Lab (postdoc);\\
 Developmental Biology and Bioinformatics; 2007-2008

    \vspace{2mm}

    \section{\mysidestyle Appointments}

    {\bf Assistant Professor}, 
Microbiology \& Molecular Genetics / Computer Science and Engineering\\
Michigan State University, 2008-2014.

    {\bf Associate Professor}, 
Population Health and Reproduction, School of Veterinary Medicine \\
University of California Davis, 2015-present.

    %____________________________________________________________________________________
    % Honours and Awards
    \section{\mysidestyle Honours and\\Awards} 

Burroughs-Wellcome Fund Computational Biology Fellowship (1999-2004).\\
Withrow Award for Teaching Excellence in Computer Science at MSU (2008-2009). \\
Woods Hole Marine Biological Laboratory Summer Fellow (2013).\\
Michigan State University / College of Natural Science Teacher-Scholar Award (2013).\\
{\bf Moore Foundation Data Driven Discovery Investigator (2014-2019).}


    \section{\mysidestyle Selected Publications}

{\em Full publication list at: http://scholar.google.com/citations?user=O4rYanMAAAAJ\\
  January 2018: 7486 citations total; h-index of 32, i10-index of 52.}

{\em Evaluating Metagenome Assembly on a Simple Defined Community with Many Strain Variants.} Awad S, Irber L, Brown CT. bioRxiv, 155358

{\em Transcriptome of the Caribbean stony coral Porites astreoides from three developmental stages}
Mansour TA, Rosenthal JJC, Brown CT, Roberson LM.
GigaScience 5 (1), 33

{\em How open science helps researchers succeed.}
McKiernan EC et al., eLife 5, e16800, July 2016.

{\em The khmer software package: enabling efficient nucleotide sequence analysis.} Crusoe MR et al., Brown CT. F1000Research. Sep 2015.

{\em Xander: employing a novel method for efficient gene-targeted metagenomic assembly.} Wang Q, Fish JA, Gilman M, Sun Y, Brown CT, Tiedje JM, Cole JR. Microbiome. August 2015.

{\em These are not the k-mers you are looking for: efficient online
  k-mer counting using a probabilistic data structure.} Zhang Q, Pell
J, Canino-Koning R, Howe AC, {\bf Brown CT}.  PLoS One. July 2014.

{\color{red} ${\bf \triangleright}$}
{\em Tackling soil diversity with the assembly of large, complex metagenomes.}
Howe AC, Jansson J, Malfatti SA, Tringe SG, Tiedje JM, {\bf Brown CT}. Accepted at PNAS, 2/2014.  preprint arXiv:1212.2832. (1 cit.)

{\em The Ribosomal Database Project: Data and Tools for High Throughput rRNA Analysis.} Cole JR, Wang Q, Fish J, Chai B, McGarrell D, Sun Y, Brown CT, Porras-Alfaro A, Kuske C, Tiedje JM.  Accepted, Nucleic Acid Res, Nov 2013.

{\em Best practices for scientific computing.} Wilson GV et al. preprint arXiv:1210.0530. Accepted PLoS Biology, October 2013. (17 cit.)

\newpage

\section{\mysidestyle Online commentaries, blogs, and social media}

Personal/professional blog at: \href{http://ivory.idyll.org/blog/}{ivory.idyll.org/blog/}.  A few selected posts (click on links): \href{http://ivory.idyll.org/blog/replication-i.html}{``Our approach to replication in computational science''}, \href{http://ivory.idyll.org/blog/thoughts-on-assemblathon-2.html}{``Thoughts on Assemblathon 2''}, \href{http://ivory.idyll.org/blog/the-future-of-khmer-2013-version.html}{``The future of khmer (2013)''}.

Twitter: \href{http://twitter.com/ctitusbrown}{@ctitusbrown}

\href{http://blogs.biomedcentral.com/bmcblog/2013/02/28/version-control-for-scientific-research/}{BioMedCentral invited blog post: ``Version control for scientific research''}

\section{\mysidestyle Selected Invitations and Meetings}

\begin{list1}

\item[] April 2013 - NIH NHGRI Education and Training committee.
\item[] March 2013 - Invited speaker at National Center for Atmospheric Research Software Engineering Assembly.
\item[] March 2013 - NSF/Moore Foundation meeting on Cyberinfrastructure for Marine 'Omics.
\item[] September 2012 - Invited speaker at Extremely Large Databases 2012 (XLDB 2012).
\item[] June 2012 - NSF BIO Centers meeting on Cyberinfrastructure Needs in BIO.

\end{list1}

    %____________________________________________________________________________________
    \section{\mysidestyle Professional\\Activities}

\begin{list1}
\item[] iPlant Scientific Advisory Board member.
\item[] Software Carpentry Scientific Advisory Board member.
\item[] Accredited Software Carpentry Instructor.
\item[] NIH Committee Member for Cloud Computing and the Human Microbiome.
\item[] Cephalopod Genome Sequencing Consortium Steering Committee, 2012-present.
\item[] Member of the Editorial Board for Open Research Computation,
Frontiers in Livestock Genomics.
\item[] Xconomist.com, invited member, Advisory Board (Michigan chapter).
\item[] BEACON NSF STC, Thrust Group co-leader (responsible for reviewing
proposals, organizing activities), 2010-2014.
\item[] Course director, 2010-present, Next-Generation Sequence Analysis for Biologists, KBS, MSU.
\item[] Course faculty, 2006-2008, Embryology Course, Woods Hole Marine Biological Laboratory.
\item[] Founder, Caltech Bioinformatics Journal Club; biology-in-python
mailing list.
\item[] Faculty advisor, Metagenomics Journal Club at MSU.
\item[] Development and maintenance of several open source bioinformatics tools, including
Cartwheel server for comparative sequence analysis, khmer k-mer software,
and screed; github.com/ctb/.
\item[] Active in open source testing community: twill, figleaf, pony-build.
\item[] Reviewer for National Science Foundation; Developmental Biology, BMC Bioinformatics, BMC Genomics, Genome Biology, Bioinformatics, PLoS One.
\end{list1}

\vspace{3cm}

\newpage

\section{\mysidestyle Teaching and Workshops}

\begin{list1}

\item[] Open Problems in Bioinformatics, CSE/MMG graduate seminar
  course (2008-2009)
\item[] Database-Backed Web Development, CSE 4xx (2008-, yearly)
\item[] Introduction to Computational Science for Evolutionary Biologists, CSE 801 (was 891) (2010-, yearly).
\item[] Analyzing Next-Generation Sequencing Data, research workshop (2010-, yearly).
\item[] Software Carpentry workshops: Scripps Research Institute (11/2012), U. Arizona (4/2013)
\item[] Instructor at Marine Biological Laboratory course on Strategies and Techniques for Analyzing Microbial Population Structures, 2012 and 2013.
\item[] Co-instructor for Workshop on Microbial Bioinformatics, 10/2013, Caltech.
\item[] Lead instructor for Workshop on mRNAseq for Biologists, and Workshop for Advanced Bioinformatics Developers, 11/2013, The Centre For Genome Analysis, Norwich, UK.
\end{list1}

\section{\mysidestyle Former Students}

Jiarong Guo (MS in Fisheries and Wildlife, 2010).  Thesis topic:
Phylogenetic analysis of annotations for uncultured bacteria.
Currently working on his PhD in Microbiology at MSU with Dr. James
Tiedje and myself.

Dr. Jason Pell (PhD in Computer Science, MSU, 2013). Thesis topic:
Efficient algorithms for the analysis of sequencing data.  Currently
working at Google.

Dr. Likit Preeyanon (PhD in Microbiology and Molecular Genetics, MSU,
2014). Thesis topic: Exploring mechanisms of genetic resistance to
Marek's Disease.  Professor (permanent position) at Mahidol
University, Thailand (2014-present).

Dr. Qingpeng Zhang (PhD in Computer Science, MSU, 2015).  Thesis
topic: A novel strategy for analyzing metagenomic sequence from many
samples.  Postdoc at Joint Genome Institute (2015).

Dr. Elijah Lowe (PhD in Computer Science, MSU, 2015).  Thesis topic:
Evolutionary developmental biology and genomics of the Molgulid
ascidians.  Postdoc at Stazione Zoologica in Naples (2014-present).

\section{\mysidestyle Current Students}

\begin{list1}
\item[] Camille Scott (CSE PhD student, 2012-2018 (expected))
\item[] Luiz Irber (CSE PhD student, 2014-2019 (expected))
\item[] Lisa Cohen (Physiology PhD student, 2015-2019 (expected)).
\item[] Taylor Reiter (Food Sciences PhD student, 2017-2021 (expected)).
\end{list1}


%________________________________________________________________________________________

\section{\mysidestyle Postdoctoral trainees}

\begin{list1}
\item[] Dr. Adina Chuang Howe (now Assistant Professor at Iowa State)
\item[] Dr. Kanchan Pavangadkar (now undergraduate advisor at MSU)
\item[] Dr. Sherine Awad (doing a second postdoc in England for visa reasons)
\item[] Dr. Tamer Mansour (current postdoctoral fellow, VetMed)
\item[] Dr. Harriet Alexander (starts faculty position at WHOI in 2018)
\item[] Dr. Daniel Standage (current postdoctoral fellow, metagenomics)
\item[] Dr. Karen Word (current postdoctoral fellow, assessment)
\item[] Dr. Phillip Brooks (current postdoctoral fellow, metagenomics)
\end{list1}

\section{\mysidestyle References}

{\em (Contact details available upon request.)}

Dr. Ewan Birney, Associate Director of the EMBL-EBI.

Professor Jonathan Eisen, University of California Davis.

Professor Paul W. Sternberg, California Institute of Technology.

Professor Billie J. Swalla, University of Washington at Seattle.

\vspace{2cm}

\end{resume}

\end{document}

%________________________________________________________________________________________
% EOF


